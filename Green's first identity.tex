\documentclass[12pt]{article}

\usepackage{amsmath}
\usepackage{amssymb}
\allowdisplaybreaks

\begin{document}
    \begin{section}{Green's First Identity}
        Green's first identity:
        $$
        \int \psi \nabla^2 \phi + \nabla \psi \cdot \nabla \phi \, dV = \oint \psi \nabla \phi \cdot d \vec S
        $$

        What I get by integrating the QG equation is
        $$
            0= \partial_t \left[ \frac12 \iint \nabla\psi\cdot\nabla\psi + F\psi^2 \, dA + \frac12 \left. \psi^2(y,t) \right|_{y=0}^L \right]
        $$

        Am I able to neglect the last term? I know I can set $\psi$ on one of the boundaries to be 0 at a certain time, but can I do it for all time? And if so, can I set $\psi$ on the other boundary to be a constant for all time, too? If I can, then that last term is just a constant and can be neglected, which gives us the exact same Hamiltonian as the doubly periodic BC's, which will give the same stability criteria.
    \end{section}

\newpage
\begin{section}{Zonal channel Boundary Conditions}
    No $x$-independence implies
    $$
        \phi = \phi(y, t), \psi = \psi(y, t)
    $$
    For any functions $u(y,t), v(y,t)$, the surface integral evaluates to
    \begin{align*}
        &\oint u \vec\nabla v \cdot d\vec S \\
        &= \lim_{X \rightarrow \infty} \frac{1}{2X} \oint_S u(y,t) v_y(y,t) \, dy + u(y,t) v_x(y,t) \, dx \\
        &= \lim_{X \rightarrow \infty} \frac{1}{2X} \oint_S u(y,t) v_y(y,t) \, dy \\
        &= \lim_{X \rightarrow \infty} \frac{1}{2X} \left[ \int_0^L u(y,t)u_y(y,t) \, dy + \int_0^L u(y,t)v_y(y,t) \, dy \right] \\
        &= 0
    \end{align*}
    If we assign $u = \psi_i$ and either $v = \dot\psi_i$ or $v = \psi_i$, we get 0. This result is used in the derivation of the Hamiltonian.
\end{section}

\newpage
\begin{section}{Doubly Periodic Boundary Conditions}
\end{section}

\end{document}
